\documentclass[TheoreticalPhy_ModB.tex]{subfiles}
\begin{document}

\chapter{Introduction}

\section{Free Fields Theories}
\subsection{Free Fields}
Here we recall the expressions of quantum  free fields.
Notice that real (i.e. physical) fields are never free because we have interactions, but using interaction picture we reconduct the problem in a simper one, where fields are described by free fields. This can be possible with a proper choice
\[
\varphi_I (x) \equiv \varphi_{\textup{free}}(x) \qquad \varphi_I \text{ = field in interacting picture}
\]

\subsubsection{Complex Scalar Field}

Euler-Lagrange equation for the scalar field:
\[(\Box + m^2) \phi = 0\]
Fourier expantion:
\[
\phi(x) = \frac{1}{(2 \pi)^{3/2}} \int \frac{d^3 k}{\sqrt{2 \omega_k}} \bigl( e^{-ikx} a(k) + e^{ikx} b^{\dagger}(k) \bigr)_{k_0 = \omega_k=\sqrt{m^2 + \vec{k}^2}} 
\]
In the real case 
\[\phi^{\dagger}(x) = \phi(x) \so a(k) = b(k)\]

\subsubsection{Dirac Spinorial Field}
Euler-Lagrange equation for the Dirac field:
\[(i \slashed{\partial} - m) \psi = 0\]
Fourier expantion:
\[
\psi(x) = \frac{1}{(2 \pi)^{2/3}} \int \frac{d^3 k}{\sqrt{2 \omega_k}} \sum_{r=1,2} \bigl( e^{-ikx} u_r (k) c_r(k) + e^{ikx} v_r(k) d^{\dagger}_r(k) \bigr)_{k_0 = \omega_k} 
\]
where $u_r(k)$ and $v_r(k)$ are respectively the $\epsilon > 0$ and $\epsilon <0$ spinors with helicity indicated by $r$.
Spinors are normalized according to
\[
\begin{cases}
\bar{u_r}(k) u_s(k) = 2m \delta_{rs}	& \bar{u_r}(k) v_s(k) = 0 \\
\bar{v_r}(k) v_r(k) = -2m \delta_{rs} 	& \bar{v_r}(k)u_s(k) = 0
\end{cases}
\]

\subsubsection{Real E-M Vector Field}
Euler-Lagrange equation for the real E-M field:
\[\partial_{\mu} F^{\mu \nu} = \Box A^{\nu} - \partial^{\nu}(\partial_{\mu} A^{\mu}) = 0\]
where 
\[F^{\mu \nu} = \partial^{\mu} A^{\nu} - \partial^{\nu} A^{\mu}\]
Fourier expantion:
\[
A^{\mu}(x) = \frac{1}{(2 \pi)^{2/3}} \int \frac{d^3 k}{\sqrt{2 \omega_k}} \sum_{\lambda=1,2} \bigl( e^{-ikx} \epsilon_{(\lambda)}^{\mu} a_{\lambda}(k) + e^{ikx} \epsilon_{(\lambda)}^{\mu \dagger}(k) a^{\dagger}_{\lambda}(k) \bigr)_{k_0 = \omega_k = \abs{\vec{k}}}\]
where polarization vectors can be chosen  as follows:
\[
\epsilon_{(1)}^{\mu} = (0, 1, 0, 0) \qquad \epsilon_{(2)}^{\mu} = (0, 0, 1, 0) 
\]
I can complexify the field substituting $a_{\lambda}^{\dagger}$ with another operator $b_{\lambda}$ (analogously to the scalar field).\\


\subsection{Fock Space of Free Fields}
\textsf{See Maggiore. See 6.1}

We impose the existence of vacuum state $\ket{0}$, with $\braket0=1$,  then using creation operators we obtain other states $(a^{\dagger})^n \ket{0}$, which are n-particles states.\\
In QFT we normalized states in a covariant way, instead of QM normalization $\int \psi^* \psi = 1$:%
\begin{subequations}
\begin{gather}
\ket{1(p)} \equiv (2 \pi)^{3/2} \sqrt{2 \omega_p}\ o^{\dagger}(p) \ket{0} \label{eqn:norm-states} \\
\braket{1(p)}{1(p')} = (2 \pi)^3 (2 \omega _p)\delta^3(p - p') \label{eqn:norm-trans} 
\end{gather}
\end{subequations}
where the term $(2 \omega_ p)\delta^3(p - p')$ is covariant under Lorentz transformations and $o^\dagger$ denotes a generic creation operator.

\subsection{Contraction of Fields with States}\label{sec:cov-norm-feynm-ruls}
If we have a state $\ket{e^-_s(p)}$ that describes an electron with momentum p and Dirac index s, then
\[\ket{e^-_s(p)} = (2 \pi)^{2/3} \sqrt{2 \omega _p}\ c^{\dagger}_s (p) \ket{0}\]
Given a field $\psi$ that describes a particle annihilation (or antiparticle creation) in the coordinate $x$ we have
\[\begin{split}
\bra0\psi(x)\ket{ e^-_s(p)}	& = \bra0  (\psi_+(x) + \psi_-(x)) \ket{ e^-_s(p)} \\
						& = \frac{(2 \pi)^{3/2}}{(2 \pi)^{3/2}} 
	\int \frac{\de^3 k}{\sqrt{2\omega_k}} e^{-ikx} \sqrt{2\omega_p} \
	\sum_{r} \bra0 c_r(k) c_s^{\dagger}(p)\ket0 u_r(k) \\
						& = \int \de^3 k \Bigl( \frac{2 \omega_p}{2 \omega_k} \Bigr)^{1/2} \
	\sum_r \delta_{rs} \delta^{(3)} (\bar{p} - \bar{k}) u_r(k) \braket00 e^{-ikx} \\
						& = e^{-ikx}u_s(p) 
\end{split}\]
where we used $\bra 0c_r(k) c_s^{\dagger}(p)\ket 0 = \bra 0\{ c_r(k), c_s^{\dagger}(p) \} \ket 0= \delta_{rs} \ \delta^3 (\bar{p} - \bar{k})$. 
The factor $e^{-ikx}$ is required for the $\delta^{(4)}$ conservation. We see that the relativistic normalization leads to Feyman rules without normalization factors: \\
\begin{equation*}
e^{-ipx}u_s(p)=
\begin{tikzpicture}[baseline=(a)]
  \begin{feynman}[medium]
	\vertex(a);
   	\vertex [right=2cm of a, dot, label={[above right=3mm of b]\(x\)}](b){\hspace {1.5mm}};
	\diagram*{
	(a) -- [fermion, momentum={[arrow shorten=0.3]\(p\)}] (b),
	};
  \end{feynman}
\end{tikzpicture}
\end{equation*}

\section{$S$-Matrix}
\subsection{Interaction Picture}
In the interaction picture, with $H = H_0 + H_{\textup{int}}$, where $H_{\textup{int}}$ is the interaction Hamiltonian
\begin{enumerate}
\item fields $\varphi_I$ evolves like in the free theory (respect to $H_0$)
\item states evolve with the following evolution operator 
	\begin{gather}
	U_I(t, t_0) \equiv e^{iH_0t} e^{-iH(t-t_0)} e^{-iH_0t_0} \notag \\
	\ket{\alpha, t} = U_I (t - t_0) \ket{\alpha, t_0}
	\qquad
	i \partial_t U_I (t-t_0) = H_I^{\textup{int}}(t) U_I(t, t_0) \notag
	\end{gather}
\item operators in interaction picture are (let $O_H(t)$ be a generic operator in Heisenberg picture and $O_I(t)$ the same operator in Interaction picture)
	\[O_I(t) = e^{iH_0t}e^{-iHt} \ O_H(t) e^{iHt} e^{-iH_0t}\]
	Notice that, in general
	\[[H_I(t), H_0] \ne 0 \ne [H_I^{\textup{int}}, H_0]\]
	and if $t \ne t'$ we also have
	\[[H_I^{\textup{int}}(t), H_I^{\textup{int}}(t')] \ne 0\]
\end{enumerate}

\subsection{$S$-matrix and States Evolution}

The \textbf{$S$-matrix} is a well defined operator defined as
\[S = \lim_{\substack{t_0 \to -\infty \\ t \to +\infty}} U_I(t, t_0)\]
We compute it by perturbation obtaining
\[\begin{split}
S 	& = T \Biggl( \exp \biggl( -i \int \de^4 x \,\mathH_I^{\textup{int}}(x) \biggr) \Biggr) \\
	& = \sum_{n=0}^{+\infty} \frac{(-i)^n}{n!} \ \int \de^4 x_1 \dots \de^4 x_n \, T\big[\mathH_I^{\textup{int}}(x_1) \dots \mathH_I^{\textup{int}}(x_n)\big]
\end{split}\]
$S$-matrix has some relevant properties:
\begin{enumerate}
\item{unitarity (since Hamiltonian is hermitian)}
\item{behaves as a scalar under Lorentz transformations, and then is an invariant quantity (notice that in general }$\mathH_I^{\textup{int}}$ is not invariant).
In the case of $\mathH_I^{\textup{int}}$ invariant (for example if, as in many theories, $\mathH_{\textup{int}} = -\mathL_{\textup{int}}$, for example in QED) is easy to prove that $S$ is invariant, since all n-th derivatives of $\exp( \int \mathH)$ are invariant, and so also $S$ is invariant.
\end{enumerate}

\subsection{$S$-matrix and transition probabilities}
Assume that there's no interaction for $x, t \to \pm\infty$.
Consider canonically normalized (CN) states $\abs{\psi} = 1$:
\[
\ket{\psi_i}_{CN} \equiv \ket{\psi(-\infty)}_{CN}
\qquad
\ket{\psi(+\infty)}_{CN} \equiv S \ket{\psi_i}_{CN}
\]
(both are free particle states).
Elements of S are in the form
\[
S_{fi}^{CN} = \prescript{}{CN}{\bra{\psi_f}} S \ket{ \psi_i}_{CN}
\]
This leads to a probabilistic interpretation of S-matrix elements. 
\begin{mdframed}[style=mybox]
The squared amplitude $\abs*{S_{fi}^{CN}}^2$ is the transition probability of $\ket{\psi_i}_{CN}$ into $\ket{\psi_f}_{CN}$. 
\end{mdframed}
Notice that the requirement $\sum_f \abs*{S_{fi}^{CN}} = 1$ is satisfied automatically. Moreover the canonical normalization is required in order to interpret such amplitude as a transition amplitude.
In the case of covariant normalization
\[
\braket{1(p)}{1(p')} = (2 \pi)^3 (2 \omega _p)\delta^3 (\vec{p} - \vec{p}')
\]
we have the following relation between matrix elements
\[
S_{fi}^{CN} = \prescript{}{CN}{\bra{\psi_f}} S \ket{\psi_i}_{CN} = \frac{\bra{\psi_f} S \ket{ \psi_i}}{\norma{\psi_i} \norma{\psi_f}} = \frac{S_{fi}}{\norma{\psi_i} \norma{\psi_f}}
\]
We can define the \textbf{Feynman Amplitude $\mathM_{fi}$} as
\[
S_{fi} = (2 \pi)^4 \delta^4 (p_i - p_f) \mathM_{fi}
\]
and it can be obtained directly starting from Feynman rules (used with the covariant normalization described in Sec. \ref{sec:cov-norm-feynm-ruls})

\section{Discrete space normalization}

Usually, in order to make arguments clearer, or to avoid problems with divergent terms in calculations, we first consider a system in a cubic box with spatial volume $V = L^3$.
At the end of computations V will be sent to infinity. Sometimes we will do something similar also for time.
For a discrete space we must use a different normalization.

In a box, the momentum of a particle is quantized
\[
p_i = \biggl( \frac{2 \pi}{L} \biggr) n_i \qquad n_i \in \mathZ
\]
and we must adopt the following rule for integrals
\[
\int \de^3 p \, f(\vec{p}) \quad \to \quad \sum_{\vec{n}} \biggr( \frac{2 \pi}{L} \biggl) f_{\vec{n}}
\qquad \vec{n} = (n_1, n_2, n_3)
\]
We must adopt also the following
\[
\delta^3 (\vec{p} - \vec{p}')
\quad \to \quad
\biggl( \frac{L}{2 \pi} \biggr)^3 \delta_{\vec{n}\vec{n}'}
\]
in this way the defining relation for the delta function works also in the discrete version
\[
\int \de^3p \, \delta^3 (\vec{p} - \vec{p}') = 1
\quad \to \quad
\sum_{\vec{n}} \biggl( \frac{2 \pi}{L} \biggr)^3 \biggl( \frac{L}{2 \pi} \biggr)^3 \delta_{\vec{n}\vec{n}'} = 1
\]
In particular
\[
\delta^3(0) \to \biggl( \frac{L}{2 \pi} \biggr)^3 
\]
When we consider also a finite amount of time we have
\[
\delta^4 (0) \to \biggl( \frac{L}{2 \pi} \biggr) \biggl( \frac{T}{2 \pi} \biggr)
\]
Using \eqref{eqn:norm-trans}, we have
\[
\braket{1(p)}{1(p)} = (2 \pi)^3 2 \omega_p \delta^3(0) = 2 \omega_p V 
\]
so normalization of states \eqref{eqn:norm-states} becomes ($o^\dagger$ is a general creation operator)
\[
\ket{1(p)} =  \sqrt{2 \omega_p V} \ket{1(p)}_{CN}
\]

\subsection{$S$-Matrix in discrete space}
Using the latter equation, $S_{fi}^{CN}$ reads
\[
\begin{split}
S_{fi}^{CN}	& = \prod_{j=1}^{n_i} \biggl( \frac{1}{2 \omega_j V} \biggr)^{1/2} \prod_{l=1}^{n_f} \biggl( \frac{1}{2\omega_l V} \biggr)^{1/2} S_{fi}\\
			& = (2 \pi)^4 \delta^4 (p_i - p_f)
				\Biggl\{ \prod_{j=1}^{n_i} \biggl( \frac{1}{2 \omega_j V} \biggr)^{1/2}
					    \prod_{l=1}^{n_f} \biggl( \frac{1}{2 \omega_l V} \biggr)^{1/2}
				\mathM_{fi} \Biggr\} \\
			& = (2 \pi)^4 \delta^4 (p_i - p_f) \mathM_{fi}^{CN}
\end{split}
\]
In the second passage we use the definition of $\mathM_{fi}$, omitting the quantization of $\delta^4$.
$\mathM_{fi}^{CN}$ is the \textbf{canonically normalized Feynman amplitude}:

\[S_{fi}=(2\pi)^4\delta^4(p_i-p_f)\mathcal M_{fi} \quad\leftrightarrow\quad S_{fi}^{CN} = (2 \pi)^4 \delta^4 (p_i - p_f) \mathM_{fi}^{CN}\]
with
\[\mathM_{fi}^{CN}=\prod_{j=1}^{n_i} \biggl( \frac{1}{2 \omega_j V} \biggr)^{1/2}
					    \prod_{l=1}^{n_f} \biggl( \frac{1}{2 \omega_l V} \biggr)^{1/2}
				\mathM_{fi}\]

\end{document}