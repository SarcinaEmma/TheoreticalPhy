\documentclass[TheoreticalPhy_ModB.tex]{subfiles}
\begin{document}

\newcommand{\lag}{\mathcal L}
\newcommand{\pde}[1]{\partial_{#1}}
\newcommand{\ipde}[1]{\partial^{#1}}
\newcommand{\mn}{{\mu\nu}}
\newcommand{\lr}{{L,R}}
\newcommand{\mcu}{\mathcal U}
\newcommand{\mcv}{\mathcal V}
\newcommand{\ckm}{V_{\text{CKM}}}
\newcommand{\pmns}{V_{\text{PMNS}}}

\chapter{The EW sector of the Standard Model}
\textsf{Schwartz sec. 29.1, 29.3, 29.5; Mandl sec. 18.3, chap. 19; Peskin chap. 20}\\

Electro-weak interactions can be (phenomenologically) be explained by introducing appropriate (renormalizable) couplings $\bar\psi\gamma^\nu\psi\nu_\mu$. Indeed
\begin{enumerate}[label=\textbullet]
\item the photon is massless, then QED can be described by an unbroken $U(1)$ gauge theory
\item the weak bosons are massive ($\sim100$GeV), and can be described by a SSB gauge theory, through the Higgs mechanism
\item phenomenologically we know that $\alpha_{EM}\neq\alpha_W$ (since $e\neq g$) so we cannot embed the 4 gauge bosons in a single group (as for example $U(2)$)
\end{enumerate}

\section{The $SU(2)_L\times U(1)_Y$ gauge sector}

Let's assume that the gauge group of the Standard Model is $\boxed{G_{SM}\equiv SU(2)_L\times U(1)_Y}$. The Yang-Mills Lagrangian for $G_{SM}$ reads
\[\boxed{\lag_{YM}=-\frac12\Tr[\hat W^{\mu\nu}\hat W_{\mu\nu}]-\frac14B^{\mu\nu}B_{\mu\nu}}\]
with following field-strength tensors
\[\begin{cases}\begin{aligned}
\hat W_{\mu\nu}&=\pde\mu\hat W_\nu-\pde\mu+ig\left[\hat W_\mu,\hat W_\nu\right]&&\quad\to\quad SU(2)\\
B_{\mu\nu}&=\pde\mu B_\nu-\pde\nu B_\mu&&\quad\to\quad U(1)
\end{aligned}\end{cases}\]
Notice that elements of the $W_{\mu\nu}$ gauge vector boson are
\[W_{\mu\nu}^a=\pde\mu W_\nu^a-\pde\nu W_\mu^a-g\epsilon^{abc}W_\mu^b W_\nu^c\]

For phenomenological reasons we have to set $U(1)_Y\neq U(1)_{EM}$,\footnote{The Noether charged associated to $U(1)_{EM}$ is the \emph{electric charge}, on the other side $U(1)_Y$ correspond to the \emph{hypercharge} $Y$ we will define in the following. } which implies
\[\begin{cases}
B_\mu\neq A_\mu\\
W_\mu^3\neq Z_\mu
\end{cases}\]
This is required by the fact that $W_\mu^3$ has only left-handed interactions, while $Z_\mu$ (which is related to the weak neutral current) has both left and right-handed couplings.

The YM Lagrangian is invariant under the local transformations given by $G_{SM}$, which are determinated by $\{T_i\}$ and $Y$, generators respectively of $SU(2)_L$ and $U(1)_Y$:
\[\begin{cases}\begin{aligned}
\Omega(x)&=e^{ig\hat\alpha(x)}\quad&&\to&&&\quad\hat\alpha(x)&=\alpha_i(x)T^i\quad\in\quad SU(2)_L\\
\omega(x)&=e^{ig'\hat\beta(x)}\quad&&\to&&&\quad\hat\beta(x)&=\beta(x)Y\quad\in\quad U(1)_Y
\end{aligned}\end{cases}\]
with the gauge fields transforming as:
\[\begin{aligned}
\Omega(x)\quad&\to\quad\begin{cases}
\hat W_\mu'=\Omega\hat W_\mu\Omega^\dagger+\frac ig(\pde\mu\Omega)\Omega^\dagger\\
B_\mu'=B_\mu\qquad(B_\mu\text{ invariant under $SU(2)_L$})
\end{cases}	\\
\omega(x)\quad&\to\quad\begin{cases}
\hat W_\mu'=\hat W_\mu\qquad(\hat W_\mu\text{ invariant under $U(1)_y$})\\
B_\mu'=\omega B_\mu\omega^\dagger+\frac i{g'}(\pde\mu W)\omega^\dagger=B_\mu-\pde\mu\beta(x)
\end{cases}\end{aligned}\]
A generic $G_{SM}$ transformation then reads
\[g_{SM}=\omega\cdot\Omega\]


\section{The Gauge-Higgs Lagrangian and SSB of the EW gauge group}

To induce the SSB mechanism we need to introduce a scalar sector with a non-trivial potential. The Lagrangian of the minimally coupled complex scalar field reads
\[\boxed{\lag_{H}=(D_\mu\phi)^\dagger(D^\mu\phi)-\lambda\p{\phi^\dagger\phi-\frac{v^2}2}^2}\]
with a covariant derivative defined in such a way that gives a coupling with $G_{SM}$: 
\[D_\mu\phi=(\partial_\mu+ig\hat W_\mu+ig'\hat B_\mu)\phi\quad\to\quad\begin{cases}\begin{aligned}\hat W_\mu&=W_\mu^i T_i\\\hat B_\mu&=B_\mu Y\end{aligned}\end{cases}\]
with $T_i$ and $Y$ the generators of $SU(2)_L\times U(1)_Y$.
Let's choose $\phi$ to be in the fundamental representation fo $SU(2)_L$ and with eigenvalue for the hypercharge $Y=\frac12$ (i.e. it is a doublet). Then
\[T_i=\frac12\sigma_i\qquad\text{and}q\qquad Y=\frac12\id_2\]
There are several way we can parametrize the complex doublet $\phi$. Having a non trivial potential let's choose the \emph{exponential notation}:
\[\phi=\begin{pmatrix}\varphi_+\\\varphi_-\end{pmatrix}\overset{(\text{SSB})}{\equiv}e^{i\hat\pi/v}\begin{pmatrix}0\\\frac{h+v}{\sqrt2}\end{pmatrix}\qquad\to\qquad\begin{cases}\hat \pi(x)=\pi_i(x)T^i\\h(x)\end{cases}\]
where $\varphi_+$, $\varphi_-$ are complex fields and we chosen the following  \emph{vacuum configuration}
\begin{equation}\label{eqn:vacuum-config-SM}
\langle\phi\rangle_0=\frac v{\sqrt2}\begin{pmatrix}0\\1\end{pmatrix}
\end{equation}
hence $\hat\pi(x)$ and $h(x)$ describes the fluctuations around $\langle\phi\rangle_0$, in particular $\pi_i(x)$ are Goldstone bosons and $h$ is the Higgs field..

All the $SU(2)_L\times U(1)_Y$ generators of the basis are broken by this choice of the vacuum
\[T_i\begin{pmatrix}0\\1\end{pmatrix}\neq0\qquad\text{and}\qquad Y\begin{pmatrix}0\\1\end{pmatrix}\neq0\]
but one particular combination of generators is unbroken, indeed
\[Q=T_3+Y=\begin{pmatrix}\frac12&0\\0&-\frac12\end{pmatrix}+\begin{pmatrix}\frac12&0\\0&\frac12\end{pmatrix}=\begin{pmatrix}1&0\\0&0\end{pmatrix}\quad\to\quad Q\begin{pmatrix}0\\1\end{pmatrix}=0\]

The spontaneous symmetry breaking induced by the scalar field with non trivial vacuum expectation valus is then 
\[SU(2)_L\times U(1)_Y\quad\to\quad U(1)_{EM}\]
and we identify $Q$ as the generator of the $U(1)_{EM}$ as we want the photon to remain massless. Indeed only unbroken symmetries corresponds to preserved symmetries after SSB, and they are related to massless bosons. 

From the representation of $Q$ we obtain that $\varphi_+$ is a charged field: $Q\begin{pmatrix}\varphi_+\\0\end{pmatrix}=+\begin{pmatrix}\varphi_+\\0\end{pmatrix}$, while $\phi_0$ is the neutral  field ($Q\begin{pmatrix}0\\\varphi_-\end{pmatrix}=0$). This also implies that $h$ is a real neutral scalar field. 

The final theory will possess an explicit gauge invariance
\[U(1)_{EM}\quad\to\quad\omega_{EM}=e^{ieQ\alpha(x)}\qquad\text{with}\quad\hat\alpha=Q\alpha\quad\text{and}\quad Q=\begin{pmatrix}1&0\\0&0\end{pmatrix}\]



The physical component $h$ corresponds to the \emph{EM neutral scalar}, indeed corresponds to the d.o.f. in the choice of the vacuum expectation value in eq.~\eqref{eqn:vacuum-config-SM}. 

Let's study the \emph{Gauge-Higgs Lagrangian} in the \emph{unitary gauge} by setting $\pi_i=0$, i.e. all Goldstone bosons are ``eaten'' by the gauge boson.


\subsubsection{i) Higgs Potential}

The \emph{Higgs potential} contains the $h(x)$ mass term and self couplings:
\[\boxed{V(\phi^\dagger\phi)=\frac\lambda4\p{(v+h)^2-v^2}^2=\frac12(2\lambda v^2)h^2+\lambda vh^3+\frac\lambda4h^4}\]
i.e. the physical scalar has mass $m_H^2=2\lambda v^2=(125\text{GeV})^2$ and we have self-interactions for the field $h$ described by the following Feynman rules:

\[
\begin{tikzpicture}[baseline=(e0)]
	\begin{feynman}
		\vertex(e0);
		\vertex[right=of e0, label={[yshift=0.1cm, font=\footnotesize]$h$}](h1);
		\vertex[above left=of e0, label={[yshift=0.1cm, font=\footnotesize]$h$}](h2);
		\vertex[below left=of e0, label={[yshift=0.1cm, font=\footnotesize]$h$}](h3);
		\diagram*{
			(h1)--[scalar](e0),
			(h2)--[scalar](e0),
			(h3)--[scalar](e0),
		};
	\end{feynman}
\end{tikzpicture}
\quad=\quad-6i\lambda v
\hspace{1.5cm}
\begin{tikzpicture}[baseline=(e0)]
	\begin{feynman}
		\vertex(e0);
		\vertex[above right=of e0, label={[yshift=0.1cm, font=\footnotesize]$h$}](h1);
		\vertex[above left=of e0, label={[yshift=0.1cm, font=\footnotesize]$h$}](h2);
		\vertex[below left=of e0, label={[yshift=0.1cm, font=\footnotesize]$h$}](h3);
		\vertex[below right=of e0, label={[yshift=0.1cm, font=\footnotesize]$h$}](h4);
		\diagram*{
			(h1)--[scalar](e0),
			(h2)--[scalar](e0),
			(h3)--[scalar](e0),
			(h4)--[scalar](e0),
		};
	\end{feynman}
\end{tikzpicture}
\quad=\quad-6i\lambda 
\]    	
where the first symmetry factors are $3!$ for the triple and $4!$ for the quartic Higgs self-interactions. 

\subsubsection{ii) Gauge-Higgs interaction}

The \emph{gauge-Higgs interactions} are in the kinetic part of the Lagrangian, given by the covariant derivative
\[(D_\mu\phi)=\frac1{\sqrt2}\begin{pmatrix}0\\\pde\mu h\end{pmatrix}+\frac{v+h}{\sqrt2}\begin{pmatrix}\frac12g(W_\mu^1-iW_\mu^2)\vspace{0.1cm}\\\frac12(g'B_\mu-gW_\mu^3)\end{pmatrix}\]
To have diagonal mass terms let's do the following field redefinitions (rotations of fields in the neutral sector):
\[W_\mu^\pm=\frac{W_\mu^1\mp iW_\mu^2}{\sqrt2}\qquad
A_\mu=\frac{gB_\mu+g'W_\mu^3}{\sqrt{g^2+g'^2}}\qquad
Z_\mu=\frac{gW_\mu^3-g'B_\mu}{\sqrt{g^2+g'^2}}
\]
where $a_\mu$ and $Z_\mu$ are physical gauge bosons. Then the Goldstone boson has benn eaten by the gauge boson and the $(D_\mu\phi)$ term reads (unitary gauge + physical bosons)
\[(D_\mu\phi)=\frac1{\sqrt2}\begin{pmatrix}0\\\pde\mu h\end{pmatrix}+\frac{v+h}{\sqrt2}\begin{pmatrix}\frac 1{\sqrt2}gW_\mu^+\\-\frac12\sqrt{g^2+g'^2}Z_\mu\end{pmatrix}\]
Notice again that the upper part is charged while the lower part is neutral under $U(1)_{EM}$. We obtained the following form for the kinetic part of the Lagrangian in the unitary gauge
\[\boxed{(D_\mu\phi)^\dagger(D^\mu\phi)=\frac12(\pde\mu h)(\partial^\mu h)+\p{1+\frac hv}^2\left\{\p{\frac{g^2v^2}4}^2W_\mu^+ W_-^\mu+\frac12\p{\frac{(g^2+g'^2)v^2}4}Z^\mu Z_\mu\right\}}\]
and contains the $W$, $Z$ interactions with $h$ and $h^2$ and mass terms for both $W_\mu^\pm$ and $Z_\mu$.
\[M_W^2=\frac{g^2v^2}4\qquad M_Z^2=\frac{(g^2+g'^2)v^2}4\]
We can express the neutral boson rotation in term of the \textbf{Weinberg angle} $\theta_w$:
\[\begin{pmatrix}A_\mu\\ Z_\mu\end{pmatrix}=\begin{pmatrix}c_w&s_w\\-s_w&c_w\end{pmatrix}\begin{pmatrix}B_\mu\\W_\mu^3\end{pmatrix}\quad\to\quad\begin{cases}c_w=\frac g{\sqrt{g^2+g'^2}}\\s_w=\frac{g'}{\sqrt{g^2+g'^2}}\end{cases}\]
where $c_w\equiv\cos(\theta_w)$ and $s_w\equiv\sin(\theta_w)$. Then the mass term can be written as
\[M_W^2=\frac{g^2v^2}4\qquad M_Z^2=\frac{g^2v^2}{4c_\omega^2}=\frac{M_W^2}{c_w^2}\]
This implies that if $A_\mu\neq B_\mu$ then $c_w\neq1$ and $M_W<M_Z$ (this is a prediction of the model). Notice that the transformation of the field we used must be a rotation, otherwise $Z_\mu$ and $A_\mu$ kinetic terms will not be canonically normalized.

Finally we can write the Feynman rules for $hVV$ and $h^2VV$ interactions:

\[
\begin{tikzpicture}[baseline=(e0)]
	\begin{feynman}
		\vertex(e0);
		\vertex[right=of e0, label={[yshift=0.1cm, font=\footnotesize]$h$}](h1);
		\vertex[above left=of e0, label={[yshift=0.1cm, font=\footnotesize]$W_\mu^+$}](h2);
		\vertex[below left=of e0, label={[yshift=0.1cm, font=\footnotesize]$W_\mu^-$}](h3);
		\diagram*{
			(h1)--[scalar](e0),
			(h2)--[boson](e0),
			(h3)--[boson](e0),
		};
	\end{feynman}
\end{tikzpicture}
\quad=\quad2i\frac{M_W^2}v\eta_{\mu\nu}
\hspace{1.5cm}
\begin{tikzpicture}[baseline=(e0)]
	\begin{feynman}
		\vertex(e0);
		\vertex[above right=of e0, label={[yshift=0.1cm, font=\footnotesize]$h$}](h1);
		\vertex[above left=of e0, label={[yshift=0.1cm, font=\footnotesize]$W_\mu^+$}](h2);
		\vertex[below left=of e0, label={[yshift=0.1cm, font=\footnotesize]$W_\mu^-$}](h3);
		\vertex[below right=of e0, label={[yshift=0.1cm, font=\footnotesize]$h$}](h4);
		\diagram*{
			(h1)--[scalar](e0),
			(h2)--[boson](e0),
			(h3)--[boson](e0),
			(h4)--[scalar](e0),
		};
	\end{feynman}
\end{tikzpicture}
\quad=\quad2i\frac{M_W^2}{v^2}\eta_{\mu\nu}
\]

\[
\begin{tikzpicture}[baseline=(e0)]
	\begin{feynman}
		\vertex(e0);
		\vertex[right=of e0, label={[yshift=0.1cm, font=\footnotesize]$h$}](h1);
		\vertex[above left=of e0, label={[yshift=0.1cm, font=\footnotesize]$Z_\mu$}](h2);
		\vertex[below left=of e0, label={[yshift=0.1cm, font=\footnotesize]$Z_\mu$}](h3);
		\diagram*{
			(h1)--[scalar](e0),
			(h2)--[boson](e0),
			(h3)--[boson](e0),
		};
	\end{feynman}
\end{tikzpicture}
\quad=\quad2i\frac{M_Z^2}v\eta_{\mu\nu}
\hspace{1.5cm}
\begin{tikzpicture}[baseline=(e0)]
	\begin{feynman}
		\vertex(e0);
		\vertex[above right=of e0, label={[yshift=0.1cm, font=\footnotesize]$h$}](h1);
		\vertex[above left=of e0, label={[yshift=0.1cm, font=\footnotesize]$Z_\mu$}](h2);
		\vertex[below left=of e0, label={[yshift=0.1cm, font=\footnotesize]$Z_\mu$}](h3);
		\vertex[below right=of e0, label={[yshift=0.1cm, font=\footnotesize]$h$}](h4);
		\diagram*{
			(h1)--[scalar](e0),
			(h2)--[boson](e0),
			(h3)--[boson](e0),
			(h4)--[scalar](e0),
		};
	\end{feynman}
\end{tikzpicture}
\quad=\quad2i\frac{M_Z^2}{v^2}\eta_{\mu\nu}
\]

\subsection{Diagonalization of the gauge bosons mass matrix and the physical bosons}

Now that we have discussed the symmetry breaking and identified the physical gauge boson one can derive the gauge self-interactions Feynman rules.
The \emph{gauge sector} in the \emph{physical basis} can be written in the following way (we set $e\equiv gs_w$, we will see that this is actually the electric charge in EM):
\[\boxed{\begin{aligned}
\lag_{YM}
&=-\frac14W_\mn^aW_a^\mn-\frac14B_\mn B^\mn\\
&=-\frac12(\pde\mu W_\nu^+-\pde\nu W_\mu^+)(\ipde\mu W_-^\nu-\ipde\nu W_-^\mu)-\frac14Z^\mn Z_\mn-\\
&\quad-\frac14F^\mn F_\mn+\left[M^2_WW_\mu^+W_-^\mu+\frac12 M_Z^2Z^\mu Z_\mu\right]+\\
&\quad+igc_w\p{Z^\mn W_\mu^+W_\nu^--W_\mn^+Z^\mu W_-^\nu+W_\mn^-Z^\mu W_+^\nu}+\\
&\quad+ie\p{F^\mn W_\mu^+W_\nu^--W_\mn^+A^\mu W^\nu_-+W_\mn^-A^\mu W_+^\nu}+\\
&\quad+\frac{g^2}2\p{W_\mu^+W_+^\mu W^-_\nu W^\nu_--W_\mu^+W_-^\mu W_\nu^+W^\nu_-}+\\
&\quad+e^2\p{A^\mu W_\mu^+ A^\nu W_\nu^--A^\mu A_\mu W_+^\nu W_\nu^-}+\\
&\quad+g^2c_w^2\p{Z^\mu W_\mu^+ Z^\nu W_\nu^--Z^\mu Z_\mu W_+^\nu W_\nu^-}+\\
&\quad+egc_w\p{W_+^\mu W_-^\nu A_\mu Z_\nu+W_-^\mu W_+^\nu A_\mu Z_\nu-2W_+^\mu W_\mu^- Z^\nu A_\nu}
\end{aligned}}\]
where the term in squared brackets is the mass term given by the gauge-Higgs interaction.

One can derive the corresponding Feynman rules for this Lagrangian. For example the trilinear gauge vertex WWZ reads

\[
\begin{tikzpicture}[baseline=(e0)]
	\begin{feynman}
		\vertex(e0);
		\vertex[right=of e0, label={[yshift=0.1cm, font=\footnotesize]$Z^\lambda$}](h1);
		\vertex[above left=of e0, label={[yshift=0.1cm, font=\footnotesize]$W_\mu^+$}](h2);
		\vertex[below left=of e0, label={[yshift=0.1cm, font=\footnotesize]$W_\nu^-$}](h3);
		\diagram*{
			(h1)--[boson, momentum={[arrow shorten=0.3, font=\footnotesize]$p_3$}](e0),
			(h2)--[boson, momentum={[arrow shorten=0.3, font=\footnotesize]$p_1$}](e0),
			(h3)--[boson, momentum'={[arrow shorten=0.3, font=\footnotesize]$p_2$}](e0),
		};
	\end{feynman}
\end{tikzpicture}
\quad=\quad
-igc_w\left[\eta_\mn(p_1-p_2)_\lambda+\eta_{\nu\lambda}(p_2-p_3)_\mu+\eta_{\lambda\mu}(p_3-p_1)_\nu\right]\]
while the quartic gauge vertex reads
\[
\begin{tikzpicture}[baseline=(e0)]
	\begin{feynman}
		\vertex(e0);
		\vertex[above right=of e0, label={[yshift=0.1cm, font=\footnotesize]$Z_\alpha$}](h1);
		\vertex[above left=of e0, label={[yshift=0.1cm, font=\footnotesize]$W_\mu^+$}](h2);
		\vertex[below left=of e0, label={[yshift=0.1cm, font=\footnotesize]$W_\nu^-$}](h3);
		\vertex[below right=of e0, label={[yshift=0.1cm, font=\footnotesize]$Z_\beta$}](h4);
		\diagram*{
			(h1)--[boson](e0),
			(h2)--[boson](e0),
			(h3)--[boson](e0),
			(h4)--[boson](e0),
		};
	\end{feynman}
\end{tikzpicture}
\quad=\quad
ig^2c_w^2\p{\eta_{\alpha\mu}\eta_{\beta\nu}+\eta_{\alpha\nu}\eta_{\beta\mu}-2\eta_{\alpha\beta}\eta_{\mu\nu}}
\]
All the other rules can be obtained by accordingly change the coupling constant ($e$, $gc_w$,etc.).

Note that we can show the unbroken $U(1)_{EM}$ symmetry of the SSB Lagrangian. We can see it in  two ways
\begin{enumerate}
\item Assigning a charge $\pm1$ to $W_\mu^\pm$ and charge 0 to $A_\mu$ and $Z_\mu$. Then all terms in the Lagrangian have total charge 0, i.e. we have a global $U(1)$ symmetry.
\item Formally one can assign the following $U(1)_{EM}$ gauge transformation properties to the gauge boson fields
\[\begin{cases}\begin{aligned}
A_\mu\quad&\to\quad A_\mu-\pde\mu\alpha\\
Z_\mu\quad&\to\quad Z_\mu\\
W_\mu^\pm\quad&\to\quad e^{\pm iq\alpha}W_\mu^\pm\\
h\quad&\to\quad h
\end{aligned}\end{cases}\]
We can notice that the $Z_\mu$ and $h$ fields are invariant under this symmetry since are neutral fields, moreover the transformation for the fields $W_\mu^\pm$ is actually a phase transformation for the complex fields.
\end{enumerate}

From the expanded Lagrangian the $U(1)_{EM}$ symmetry looks cumbersome due to the presence of couplings with derivative terms. One should instead collect terms together defining a covariant derivative for the $W_\mu^\pm$ fields
\[\begin{cases}
\mathcal D_\mu W_\nu ^\pm\equiv(\pde\mu\pm iqA_\mu)W_\nu^\pm\\
W_\mn^\pm=\mathcal D_\mu W_\nu^\pm-\mathcal D_\nu W_\mu^\pm\mp igc_w(Z_\mu W_\nu ^\pm-Z_\nu W_\mu^\pm)
\end{cases}\]
We define $\mathcal D_\mu$ to be the \emph{covariant derivative for $W_\mu^\pm$ fields}. 

Note that in the previous Lagrangian we set $e\equiv gs_w$ to assign charge $\pm1$ to the $W_\mu^\pm$ fields. From this definition one can rewrite the $s_w$ and $c_w$ of the neutral sector rotation as
\[\begin{aligned}
s_w&\equiv\frac{g'}{\sqrt{g^2+g'^2}}=\frac eg\\
c_w&\equiv\frac{g}{\sqrt{g^2+g'^2}}=\frac e{g'}
\end{aligned}
\quad\to\quad e\equiv\frac{gg'}{\sqrt{g^2+g'^2}}\]
so finally one has
\[e=gs_w=g'c_w\]
Moreover we found an explicit relation between the couplings $e$, $g$ and $g'$ (one can use only two of them to describe the theory).
We will verify in the following that this definition is consistent with the fermionic charge (EM) assignment. 




\section{The gauge-fermion sector}

We have defined QED and QCD as vector like interactions, i.e. $\bar\psi\gamma^\mu\psi V_\mu$. When discussing weak interactions we saw that left and right chiralities have different couplings with weak bosons, i.e. are described by chiral currents in the form $\bar\psi(c_L\gamma^\mu_L+c_R\gamma^\mu_R)\psi V_\mu$ with $c_L\neq c_R$.
In other words we saw that weak interactions are a chiral theory. 
Then in building the SM interactions between fermions and gauge boson we have to assume
\begin{enumerate}
\item $SU(2)_L$ acts only on the left chiral fermionic component (i.e. charged currents are only left handed)
\item $U(1)_Y$ acts on both left and right chiral components (but in the general the action on the left component is different from the action on the right component)
\end{enumerate}
As left fermions transform under $SU(2)$ while right fermions do not, this gives a requirement on the representation of these fields:
\begin{enumerate}
\item Left fermions are doublets of $SU(2)$ (i.e. are in the fundamental representation)
\item Right fermions are singlets of $SU(2)$ (i.e. are in the trivial representation)
\end{enumerate}

Let's consider the first family of fermions containing the lightest fermion of each type
\[\text{leptons }=(\nu_e,e)\hspace{1.5cm}\text{quarks }=(u,d)\]
Recall that families differs only because of masses. 

Let's assign the following quantum numbers (i.e. charges) to the first fermionic family:

\begin{table}[H]
\centering
\begin{tabular}{@{}||c||c|c|c|c||@{}}
\toprule
& $SU(2)$ & $T_3$ & $U(1)_Y$ & $Q=T_3+Y$ \\ \midrule\midrule
$l_L\equiv\begin{pmatrix}\nu_e\\e\end{pmatrix}_L$ & 2 & $\begin{matrix}+1/2\\-1/2\end{matrix}$ & -1/2 & $\begin{matrix}0\\-1\end{matrix}$ \\\midrule
$\nu^e_R$ & 1 & 0 & 0 & 0 \\\midrule
$e_R$ & 1 & 0 & -1 & -1 \\\midrule
$q_L\equiv\begin{pmatrix}u\\d\end{pmatrix}_L$ & 2 & $\begin{matrix}+1/2\\-1/2\end{matrix}$ & +1/6 & $\begin{matrix}+2/3\\-1/3\end{matrix}$ \\\midrule
$u_R$ & 1 & 0 & +2/3 & +2/3 \\ \midrule
$d_R$ & 1 & 0 & -1/3 & -1/3 \\ \bottomrule
\end{tabular}
\end{table}
%
\noindent
where $T_3$ is the diagonal generator of $SU(2)_L$. Hence all right-handed fermions have $T_3=0$, in this way they are left invariant under $SU(2)_L$.  In general quantum numbers are fixed watching how these particles interacts with gauge bosons, moreover hypercharge and isospin eigenvalues (i.e. $T_3$ and $Y$) are fixed in order to satisfy $Q=T_3+Y$. In this way $Q=T_3+Y$ has the same definition of the generator of $U(1)_{EM}$ we gave before. For this reason we gave zero $Q$ eigenvalue for the neutrino. Notice that both $\nu_e$ and $e$ must have same isospin eigenvalue in order to preserve gauge invariance, same statement holds also for $u$ and $d$. For the adjoint field $\bar\psi$ the quantum numbers are the same as for $\psi$ but multiplied by $-1$. 

Knowing all the fermionic charges we can couple fermions to gauge bosons via minimal coupling. Each fermion component has its own $D_\mu$ dependency on its charges:
\[\begin{aligned}
D_\mu l_L &=\p{\pde\mu+igW_\mu^aT_a+ig'B_\mu\p{-\frac12}\id}\begin{pmatrix}\nu_e\\e\end{pmatrix}_L\\
D_\mu q_L &=\p{\pde\mu+igW_\mu^aT_a+ig'B_\mu\p{+\frac16}\id}\begin{pmatrix}u\\d\end{pmatrix}_L\\
D_\mu \nu_R &=\pde\mu\nu_R\\
D_\mu e_R&=\p{\pde\mu+ig'B_\mu\p{-1}\id}e_R\\
D_\mu u_R&=\p{\pde\mu+ig'B_\mu\p{\frac23}\id}u_R\\
D_\mu d_R&=\p{\pde\mu+ig'B_\mu\p{-\frac13}\id}d_R
\end{aligned}\]
We can see that neutrinos does not interact with weak gauge bosons. 

The \emph{fermionic-gauge Lagrangian} then can be rewritten using covariant derivatives:
\[\boxed{
\lag_F=\bar\psi_i(i\slashed D)\psi_i
=\bar l_Li\slashed D l_L+\bar q_Li\slashed D q_L+\bar\nu_Ri\slashed\partial\nu_R+\bar e_Ri\slashed De_R+\bar u_Ri\slashed Du_R+\bar d_Ri\slashed Dd_R
}\]
and contains both kinetic part and gauge interactions.
It is invariant under a $SU(2)_L\times U(1)_Y$ gauge transformations $\Omega\cdot\omega$:
\[SU(2)_L\quad\to\quad\begin{cases}L'=\Omega L\\e_R'=e_R\end{cases}
\hspace{1.5cm}
 U(1)_Y\quad\to\quad\begin{cases}L'=\omega_L L\\e_R'=\omega_Re_R\end{cases}\]

In terms of the physical gauge bosons (i.e. using mass eigenvectors) we have
\[W_\mu^\pm=\frac{W_\mu^1\mp iW_\mu^2}{\sqrt2}\qquad\text{with}\qquad T_\pm=T_1\pm iT_2\]
and
\[\begin{pmatrix}A_\mu\\ Z_\mu\end{pmatrix}=\begin{pmatrix}c_w&s_w\\-s_w&c_w\end{pmatrix}\begin{pmatrix}B_\mu\\W_\mu^3\end{pmatrix}\]
so the Lagrangian reads
\[\boxed{\begin{aligned}
\lag_F
&=\lag_{\text{KIN}}+\frac g{\sqrt2}\left\{(\bar l_L\gamma^\mu T_+l_L+\bar q_L\gamma^\mu T_+q_L)W_\mu^++\text{h.c.}\right\}+\\
&\quad+\Big\{\bar l_L\gamma^\mu(gs_wT_3+g'c_wY_{l_L})l_L+\bar q_L\gamma^\mu(gs_wT_3+g'c_wY_{q_L})q_L+\\
&\quad\quad+\bar e_R\gamma^\mu(g'c_wY_{e_R})e_R+\bar u_R\gamma^\mu(g'c_wY_{u_R})u_R+\bar d_R\gamma^\mu(g'c_wY_{d_R})d_R\Big\}A_\mu\\
&\quad+\Big\{\bar l_L\gamma^\mu(gc_wT_3+g's_wY_{l_L})l_L+\bar q_L\gamma^\mu(gc_wT_3+g's_wY_{q_L})q_L+\\
&\quad\quad+\bar e_R\gamma^\mu(-g's_wY_{e_R})e_R+\bar u_R\gamma^\mu(-g's_wY_{u_R})u_R+\bar d_R\gamma^\mu(-g's_wY_{d_R})d_R\Big\}Z_\mu
\end{aligned}}\]
where now the couplings with the photon has to reconstruct the EM charge
\[\begin{cases}\begin{aligned}
gs_wT_3+g'c_wY&=eQ&&\qquad\text{for the left component}\\
g'c_wY&=eQ&&\qquad\text{for the right component}
\end{aligned}\end{cases}\]
For example for the leptons one has (use quantum numbers shown in the previous table)
\[\begin{cases}
gs_w\begin{pmatrix}1/2&0\\0&1/2\end{pmatrix}+g'c_wY_L\begin{pmatrix}1&0\\0&1\end{pmatrix}=e\begin{pmatrix}0&0\\0&-1\end{pmatrix}\\
g'c_wY_{e_R}=-e\\
g'c_wY_{\nu_R}=0
\end{cases}\]
that with the following assignments for Y (notice $Y_{\nu_L}+Y_{e_L}=-1/2$ as shown in the table)
\[Y_{\nu_L}=-\frac12\qquad Y_{e_L}=-1/2\qquad Y_{\nu_R}=0\qquad Y_{e_R}=-1\]
gives the following relations
\[\begin{cases}
gs_w=g'c_w\\
g'c_w=e\\
\frac12(gs_w+g'c_w)=e
\end{cases}\]
Then in order to recover the EM charge one has
\[gs_w=g'c_w=e\]
hence previous assignments are consistent.  

Note that assuming left fermions in the fundamental representation and right fermions in the trivial representation, from the previous equations one recover the \emph{hypercharge} assignment (compatible with EM)
\[Q=T_3+Y\]

\subsection{Couplings with fermions: EW, Charged and Neutral currents}

By using previous definitions the fermionic Lagrangian can be written
\[\boxed{\lag_F=\lag_{\text{KIN}}+\frac g{\sqrt2}(Y_{\text{CC}}^\mu W_\mu^++\text{h.c.})+eJ_{\text{EM}}^\mu A_\mu+\frac g{c_w}J_Z^\mu Z_\mu}\]
with the following currents
\[\begin{aligned}
J_{\text{CC}}^\mu&=\bar\nu_L\gamma^\mu e_L+\bar u_L\gamma^\mu d_L\\
J_{\text{EM}}^\mu&=\sum_{i}q_i\bar\psi_i\gamma^\mu\psi_i&&&\begin{pmatrix}\psi_i\\q_i\end{pmatrix}&=\begin{pmatrix}e&u&d\\-1&2/3&-1/3\end{pmatrix}\\
J_Z^\mu&=\sum_i\bar\psi_i(c_L^i\gamma^\mu_L+c_R^i\gamma_R^\mu)\psi_i\hspace{0.7cm}&&&\psi_i&=(\nu,e,u,d)
\end{aligned}\]
where we defined the coupling constants
\[c_L^i=t_3^i-s_w^2q_i\hspace{1.5cm}c_R^i=-s_w^2q_i\]
with $q_i$ charges of $\psi_i$ and $t_3^i$ eigenvalues of $T_3$. As we required we have $c_L\neq c_R$ for chiral couplings and $c_L=c_R$ for vector couplings. 

This Lagrangian is exactly the same as IVB. Now we can obtain the boson-fermion couplings in terms of the $SU(2)\times U(1)$ quantum numbers. 

Notice that fermionic mass terms are missing, indeed fermion mass terms are forbidden in a chiral theory by gauge invariance:
\begin{enumerate}
\item In vector-like gauge theories like QED ($U(1)_{\text{EM}}$) or QCD ($SU(3)_C$) mass terms are allowed: if we start from a Lagrangian ($\bar\psi\slashed A\psi=\bar\psi_R\slashed A\psi_R+\bar\psi_L\slashed A\psi_L$)
\[\lag_{\text{vector}}=\bar\psi(i\slashed D-M)\psi\]
the theory is invariant under gauge transformations ($\Omega\in$ gauge group)
\[\psi'=\Omega\psi\hspace{1.5cm}(D_\mu\psi)'=\Omega(D_\mu\psi)\]
indeed
\[\lag'_{\text{vector}}=\bar\psi'i(\slashed D\psi)'-M\bar\psi'\id\psi'=\bar\psi(i\slashed D-M\id)\psi=\lag_{\text{vector}}\]
\item In chiral gauge theories like $SU(2)_L\times U(1)_Y$ mass terms are forbidden: if we take a generic chiral Lagrangian with a mass term for $\psi=(\nu,e)$ and covariant derivatives $D^L_\mu=\partial_\mu+ig\hat W_\mu$ and $D^R_\mu=\partial_\mu$:
\[\lag_{\text{chiral}}=\bar\psi_Li\slashed D_L\psi_L+\bar\psi_Ri\slashed D_R\psi_R-M(\bar\psi_L\psi_R+\bar\psi_R\psi_L)\]
and for simplicity we consider only a $SU(2)_L$ transformation ($\Omega\in SU(2)_L$)
\[\psi_L=\begin{pmatrix}\nu_L\\e_L\end{pmatrix}\quad\to\quad\psi_L'=\Omega\psi_L
\hspace{1.5cm}
\psi_R=\begin{pmatrix}\nu_R\\e_R\end{pmatrix}\quad\to\quad\psi_R'=\psi_R\]
then the Lagrangian becomes
\[\begin{aligned}
\lag'_{\text{chiral}}&=\bar\psi_L'i(\slashed D_L\psi_L)'+\bar\psi_Ri\slashed \partial\psi_R-M(\bar\psi_L'\psi_R+\bar\psi_R\psi_L')\\
&=\bar\psi_Li\slashed D_L\psi_L+\bar\psi_Ri\slashed\partial\psi_R-M(\bar\psi_L\Omega^\dagger\psi_R+\bar\psi_R\Omega\psi_L)
\end{aligned}\]
and the mass term is clearly not invariant.
\end{enumerate}

\section{The Higgs-fermion sectors and mass terms for chiral fermions (1 family)}

We can give a mass term to fermions by using the Higgs SSB mechanism, coupling fermions to $\phi$. To make an $SU(2)_L\times U(1)_Y$ coupling between SM fermion and Higgs we have to match the $SU(2)_L$ and $U(1)_Y$ charges:
\[\phi=(2,1/2)\quad\begin{matrix}l_L=(2,-1/2)&e_R=(1,-1)&\nu_R=(1,0)\\q_L=(2,1/6)&u_R=(1,2/3)&d_R=(1,-1/3)\end{matrix}\]

In particular we use a Lagrangian containing \textbf{Yukawa interactions} i.e. with interacting terms which involves two fermions and one scalar. In this way we obtain the following gauge invariant Lagrangian
\[\boxed{\lag_{\text{YUK}}=-y_u\bar q_L\tilde \phi u_R-y_d\bar q_L\phi d_R-y_e\bar l_L\phi e_R-y_\nu\bar l_L\tilde\phi\nu_R}\]
where we have defined
\[\tilde\phi=i\sigma_2\phi^*=\frac{v+h}{\sqrt2}\begin{pmatrix}1\\0\end{pmatrix}\quad\to\quad\tilde\phi=(2,-1/2)\]


To check the $SU(2)_L\times U(1)_Y$ invariance one has to check the invariance of $\lag_{\text{YUK}}$ under both these sets of transformation for fields (verify the invariance of the Lagrangian as exercize):
\[\begin{aligned}
SU(2)_L\quad&\to\quad\begin{cases}\begin{matrix}q_L'=\Omega q_L&l_L'=\Omega l_L\\\phi'=\Omega\phi&\tilde\phi'=\Omega\tilde\phi\end{matrix}\end{cases}\\
U(1)_Y\quad&\to\quad\begin{cases}\begin{cases}\begin{matrix}q_L'=\omega_{1/6} q_L&l_L'=\omega_{-1/2} l_L\\\phi'=\omega_{1/2}\phi&\tilde\phi'=\omega_{-1/2}\tilde\phi\end{matrix}\end{cases}\\\begin{cases}\begin{matrix}\nu_R'=\nu_R&e_R'=\omega_{-1} e_R\\u_R'=\omega_{2/3}u_R&d_R'=\omega_{-1/3}d_R\end{matrix}\end{cases}\end{cases}
\end{aligned}\]
Notice that the sum of both $SU(2)_L$ and $U(1)_Y$ eigenvalues in each term of the previous Lagrangian gives a zero sum, for instance the sum of $U(1)_Y$ eigenvalues in $\bar q_L\tilde \phi u_R$ gives $-1/6-1/2+2/3=0$. This ensures the invariance under $U(1)_Y$ transformations we just described. 

Note that in the original definition of the SM $\nu_R$ was not introduced, since it has no effect in gauge interactions ($\nu_R=(1,0)$). Actually according to the Yukawa Lagrangian in principle it can have Yukawa interactions. 

When the Higgs field acquires a vacuum expectation value, in the unitary gauge we have
\[\phi=\frac{v+h}{\sqrt2}\begin{pmatrix}0\\1\end{pmatrix}\hspace{1.5cm}\tilde\phi=\frac{v+h}{\sqrt2}\begin{pmatrix}1\\0\end{pmatrix}\]
and the Yukawa Lagrangian reads
\[\lag_{\text{YUK}}=-\sum m_i\p{1+\frac hv}\p{\bar\psi_L^i\psi_R^i+\bar\psi_R^i\psi_L^i}\]
for $\psi_i=(\nu_e,e,u,d)$. In this way we obtained a mass term for each fermion field:
\[m_i\equiv \frac{y_iv}{\sqrt2}\]
and is called \textbf{Yukawa mass term}.

The Feynman rules associated to this interaction are very symple:

\[
\begin{tikzpicture}[baseline=(e0)]
	\begin{feynman}
		\vertex(e0);
		\vertex[left=of e0, label={[yshift=0.1cm, font=\footnotesize]$h$}](h1);
		\vertex[above right=of e0, label={[yshift=0.1cm, font=\footnotesize]$f_i$}](f1);
		\vertex[below right=of e0, label={[yshift=0.1cm, font=\footnotesize]$\bar f_i$}](f2);
		\diagram*{
			(h1)--[scalar](e0),
			(e0)--[fermion](f1),
			(f2)--[fermion](e0),
		};
	\end{feynman}
\end{tikzpicture}
\quad=\quad i\frac{m_i}v=i\frac{y_i}{\sqrt2}\]

\section{Summary of the $SU(2)_L\times U(1)_Y$ Lagrangian for 1 family}

\begin{enumerate}[label=(\arabic*)]
\item The (EW) SM Lagrangian describes the interactions between \emph{gauge bosons}, \emph{fermions} and \emph{Higgs scalars}:
\begin{enumerate}
\item $SU(2)_L\times U(1)_Y$ gauge bosons: $W_\mu^\pm$, $Z_\mu$, $A_\mu$
\item 1 family of fermions: $\begin{pmatrix}\nu_e\\e\end{pmatrix}_L$, $\nu_R$, $e_R$, $\begin{pmatrix}u\\d\end{pmatrix}_L$, $u_R$, $d_R$
\item Higgs complex scalar doublet $\phi=\begin{pmatrix}\varphi_+\\\varphi_-\end{pmatrix}=e^{i\hat\pi/v}\begin{pmatrix}0\\\frac{v+h}{\sqrt2}\end{pmatrix}$
\end{enumerate}
\[\boxed{\lag_{\text{SM}}=\lag_{\text{YM}}+\lag_{\text{H}}+\lag_{\text{F}}+\lag_{\text{YUK}}}\]

\item The Lagrangian is $SU(2)_L\times U(1)_Y$ gauge invariant

\item The non-trivial potential force the Higgs to acquire a non-vanishing vacuum expectation value $\langle\phi\rangle_0=\frac{v}{\sqrt2}\begin{pmatrix}0\\1\end{pmatrix}$ that breaks spontateously the gauge symmetry
\[SU(2)_L\times U(1)_Y\quad\overset{\text{SSB}}{\longrightarrow}\quad U(1)_{\text{EM}}\]
This gives the EM charge described by the generator $Q=\begin{pmatrix}1&0\\0&0\end{pmatrix}$. 

\item The non-vanishing vacuum expectation value is the origin of all masses of the SM fields
\begin{enumerate}
\item Weak bosons from $(D_\mu\phi)^\dagger(D^\mu\phi)$: $M_W^2=\frac{g^2v^2}4$ and $M_Z^2=\frac{(g^2+g'^2)v^2}4$
\item The Higgs mass from the Higgs potential $V(\phi^\dagger\phi)$: $m_H^2=2\lambda v^2$
\item The fermion masses from $\lag_{\text{YUK}}$: $m_i=\frac{y_iv}{\sqrt2}$
\end{enumerate}

\end{enumerate}




\section{The 3-families case and general $3\times3$ Yukawa sector and flavor violation in the CC sector: the $V_{\text{CKM}}$ and the $V_{\text{PMNS}}$ matrices}

In the previous sections we have discussed all the SM construction for 1 family (the lightest one):
\[q_L^1=\begin{pmatrix}u\\d\end{pmatrix}_L\qquad\begin{matrix}u_R\\d_R\end{matrix}\qquad l_L^1=\begin{pmatrix}\nu_e\\e\end{pmatrix}_L\qquad\begin{matrix}\nu_{e, R}\\e_R\end{matrix}\]
In nature exist other 2 replicas of this family structure, which differs only by the masses of particles
\[q_L^2=\begin{pmatrix}c\\s\end{pmatrix}_L\qquad\begin{matrix}c_R\\s_R\end{matrix}\qquad l_L^2=\begin{pmatrix}\nu_\mu\\\mu\end{pmatrix}_L\qquad\begin{matrix}\nu_{\mu, R}\\\mu_R\end{matrix}\]
\[q_L^3=\begin{pmatrix}t\\b\end{pmatrix}_L\qquad\begin{matrix}t_R\\b_R\end{matrix}\qquad l_L^3=\begin{pmatrix}\nu_\tau\\\tau\end{pmatrix}_L\qquad\begin{matrix}\nu_{\tau, R}\\\tau_R\end{matrix}\]
There are some theoretical hints about the need of a complete family (\emph{anomaly}) but why we have exactly three families is an open question. 

To deal with 3 families we have to add a 3 dimensional flavor structure:
\[\begin{aligned}
u_\lr&&\to&&U_\lr&=\begin{pmatrix}u\\c\\t\end{pmatrix}_\lr
&&\hspace{1cm}&&
d_\lr&&\to&&&D_\lr&=\begin{pmatrix}d\\s\\b\end{pmatrix}_\lr\\
e_\lr&&\to&&E_\lr&=\begin{pmatrix}e\\\mu\\\tau\end{pmatrix}_\lr
&&\hspace{1cm}&&
\nu_\lr&&\to&&&N_\lr&=\begin{pmatrix}\nu_e\\\nu_\mu\\\nu_\tau\end{pmatrix}_\lr\\
q_L&&\to&&Q_L&=\begin{pmatrix}U\\D\end{pmatrix}_L
&&\hspace{1cm}&&
l_L&&\to&&&L_L&=\begin{pmatrix}N\\E\end{pmatrix}_L
\end{aligned}\]

\subsubsection{The gauge-fermion Lagrangian}

The gauge-fermions Lagrangian with the flavor structure now reads
\[\boxed{
\lag_F=\lag_{\text{KIN}}+\frac g{\sqrt2}(J_{CC}^\mu W_\mu^++\text{h.c.})+\frac g{c_w}J_{NC}^\mu Z_\mu+eJ_{\text{EM}}^\mu A_\mu
}\]
where we define following currents
\[\begin{aligned}
J_{\text{CC}}^\mu&=\bar U_L\gamma^\mu\id D_L+\bar N_L\gamma^\mu\id E_L\\
J_{\text{EM}}^\mu&=q_U\bar U\gamma^\mu\id U+q_D\bar D\gamma^\mu D+q_E\bar E\gamma^\mu\id E\\
J_Z^\mu&=\bar U(c_L^u\gamma^\mu_L+c_R^u\gamma_R^\mu)\id U+\bar D(c_L^d\gamma^\mu_L+c_R^d\gamma_R^\mu)\id D+\bar N(c_L^\nu\gamma^\mu_L+c_R^\nu\gamma_R^\mu)\id N+\bar E(c_L^e\gamma^\mu_L+c_R^e\gamma_R^\mu)\id E
\end{aligned}\]
We define the \textbf{flavor (interaction) basis} the basis where the interactions are diagonal. We assumed \emph{universal couplings}, i.e. are family independent. With these definitions the interaction gauge-fermion Lagranian has an $SU(3)$ global symmetry. 

\subsubsection{The Higgs-fermion Lagrangian}

The Higgs-fermion Lagrangian with the flavor structure now reads
\[-\lag_{\text{YUK}}=+\bar Q_L\tilde\phi Y_uU_R+\bar Q_L\phi Y_dD_R+\bar L_L\phi Y_eE_R+\bar L_L\tilde\phi Y_\nu N_R+\text{h.c.}\]
In the interaction (flavor) basis the \textbf{Yukawa matrices} $Y_i$ are arbitrary $3\times 3$ complex matrices.

Remember that fermion masses are proportional to the Yukawa couplings Y, indeed if we set $\phi=\frac{v+h}{\sqrt2}\begin{pmatrix}0\\1\end{pmatrix}$ then:
\[M_U=\frac{Y_uv}{\sqrt2}\qquad M_D=\frac{Y_dv}{\sqrt2}\qquad M_N=\frac{Y_\nu v}{\sqrt2}\qquad M_E=\frac{Y_ev}{\sqrt2}\]
In general the mass matrices are not to be assumed diagonal in the flavor basis. According to \textbf{CKM ansatz}, \textit{all sources of $SU(3)$ flavor violation are encoded in the Yukawa matrices}. With an appropriate rotation of the fields we can go to the mass basis. 

\subsubsection{Diagonalization of the mass matrices}

Even if sometimes is useful to work in the flavor basis, the physical basis is where the physical observables (propagators) has to be calculated. This means that physical states are given by mass eigenvalues. 

A general $3\times 3$ matrix can be always diagonalized by a biunitary transformation. Given $M$ we can always build two different Hermitian matrices $MM^\dagger$, $M^\dagger M$ and diagonalize them with unitary transformations
\[\begin{cases}
MM^\dagger\quad\to\quad \mcu^\dagger(MM^\dagger)\mcu=M^2_{\text{diag}}\\
M^\dagger M\quad\to\quad \mcv^\dagger(M^\dagger M)\mcv=M^2_{\text{diag}}
\end{cases}\]
As obviously the eigenvalues are the same (real) and we can write ($\mcu\mcu^\dagger=\mcv\mcv^\dagger=\id$)
\[M^2_{\text{diag}}=(\mcu^\dagger M \mcv)(\mcv^\dagger M^\dagger \mcu)=(\mcv^\dagger M^\dagger \mcu)(\mcu^\dagger M \mcv)=M_{\text{diag}}^\dagger M_{\text{diag}}=M_{\text{diag}}M_{\text{diag}}^\dagger\]
hence we can set (the other choice is exactly equivalent)
\[M_{\text{diag}}\equiv \mcu^\dagger M \mcv\hspace{1.5cm}M_{\text{diag}}^\dagger\equiv\mcv^\dagger M^\dagger \mcu\]

Now let's rotate all fermion triplets in flavor space by appropriate $SU(3)$ elements $R_\lr,S_\lr,T_\lr,X_\lr\in SU(3)$ (in general we have to treat differently left and right components since we are considering a chiral theory)
\begin{equation}\label{eqn:rotat-flavor-mass}
\begin{matrix}
	\begin{cases}
		\begin{matrix}
			U_L'=R_LU_L & D_L'=S_LD_L\\
			U_R'=R_RU_R & D_R'=S_RD_R
		\end{matrix}
	\end{cases}
	&
	\begin{cases}
		\begin{matrix}
			N_L'=T_LN_L & E_L'=X_LE_L\\
			N_R'=T_RN_R & E_R'=X_RE_R
		\end{matrix}
	\end{cases}
\end{matrix}
\end{equation}
		
By these rotations the Yukawa Lagrangian becomes
\[\boxed{\begin{aligned}
-\lag_{\text{YUK}}&=\left\{\bar D_L'S_L^\dagger M_DS_RD_R'+\bar U_R'R_L^\dagger M_UR_RU_R'+\bar E_L'X_L^\dagger M_EX_RE_R'+\bar N_L'T_L^\dagger M_NT_RN_R'+\text{h.c.}\right\}\p{1+\frac hv}\\
&=\left\{\bar D_L'M_D^{\text{diag}}D_R'+\bar U_R'M_U^{\text{diag}}U_R'+\bar E_L'M_E^{\text{diag}}E_R'+\bar N_L'M_N^{\text{diag}}N_R'+\text{h.c.}\right\}\p{1+\frac hv}
\end{aligned}}\]
where we introduced the diagonal mass matrices
\[\begin{matrix}
	M_U^{\text{diag}}\equiv R_L^\dagger M_UR_R =
	\begin{pmatrix}m_u&0&0\\0&m_c&0\\0&0&m_t\end{pmatrix}
	&
	M_D^{\text{diag}}\equiv S_L^\dagger M_DS_R =
	\begin{pmatrix}m_d&0&0\\0&m_s&0\\0&0&m_b\end{pmatrix}
	\vspace{0.2cm}\\
	M_N^{\text{diag}}\equiv T_L^\dagger M_NT_R =
	\begin{pmatrix}m_{\nu_e}&0&0\\0&m_{\nu_\mu}&0\\0&0&m_{\nu_\tau}\end{pmatrix}
	&
	M_E^{\text{diag}}\equiv X_L^\dagger M_EX_R =
	\begin{pmatrix}m_e&0&0\\0&m_\mu&0\\0&0&m_\tau\end{pmatrix}
\end{matrix}\]

\subsubsection{The gauge-fermion interaction Lagrangian}

The gauge-fermion interaction Lagrangian has also to be rotated and written in the mass basis. 

Notice that the neutral current sector is not affected by the change of basis eq.~\eqref{eqn:rotat-flavor-mass}. Indeed all the pieces are in the form 
\[c_L\bar U_L\gamma^\mu U_L+c_R\bar U_R\gamma^\mu U_R+\dots\]
hence are transformed into 
\[c_L\bar U'_L\gamma^\mu U'_L+c_R\bar U'_R\gamma^\mu U'_R+\dots\]
Hence neutral currents are diagonal and universal in the mass basis. The neutral sector has an $SU(3)_F$ symmetry, i.e. flavor is not broken by neutral currents. 

The charged current sector however is not left diagonal by the change of basis as it involves different fields:
\[\lag_{CC}=\frac g{\sqrt2}(\bar D_L\gamma^\mu D_L+\bar E_L\gamma^\mu E_L)W_\mu^++\text{h.c.}\]
and in the mass basis reads
\[\boxed{\begin{aligned}
\lag_{CC}&=\frac g{\sqrt2}(\bar D_L'\gamma^\mu(S_LR_L^\dagger)U_L'+\bar E'_L\gamma^\mu(X_LT_L^\dagger)N_L)W_\mu^++\text{h.c.}\\
&=\frac g{\sqrt2}(\bar D_L'\gamma^\mu \ckm^\dagger U_L'+\bar E'_L\gamma^\mu\pmns^\dagger N_L)W_\mu^++\text{h.c.}
\end{aligned}}\]
where we defined the following $3\times 3$ unitary matrices\footnote{CKM=Cabibbo-Kobayashi-Maskawa \\PMNS=Pontecorvo-Maki-Nakagawa-Sakata}
\[\ckm\equiv R_LS_L^\dagger\hspace{1.5cm}\pmns\equiv T_LX_L^\dagger\]
which in general are not diagonal in the mass basis. These matrices contain all the flavor (and CP) violation of the SM. In the neutral sector there are no flavor changing couplings, only flavor violation appears in the charged sector. 

The associated Feynman rules read, for instance
\[\begin{tikzpicture}[baseline=(e0)]
	\begin{feynman}
		\vertex(e0);
		\vertex[right=of e0, label={[yshift=0.1cm, font=\footnotesize]$D_j$}](d);
		\vertex[left=of e0, label={[yshift=0.1cm, font=\footnotesize]$U_i$}](u);
		\vertex[below =of e0, label={[xshift=0.45cm, font=\footnotesize]$W_\mu^\pm$}](w);
		\diagram*{
			(u)--[fermion](e0),
			(e0)--[fermion](d),
			(w)--[boson](e0),
		};
	\end{feynman}
\end{tikzpicture}
\quad=\quad i\frac g{\sqrt2}V_{ij}^*\gamma_L^\mu
\]
where $u_i$ and $d_j$ are elements of $D$ and $U$ respectively, with indices $i$ and $j$. Reverting the role of $U$ and $D$ in the diagram the matrix element $V_{ij}^*$ must be substituted with $V_{ij}$. 

Notice that the CC sector brokes both $SU(3)$ and $U(1)^3$ flavor symmetries. There is only a residual global $U(1)$ symmetry, which corresponds to the conservation of the \textbf{Baryon number}, which is unbroken by $\ckm$ and $\pmns$, as we will see later. Flavor changing couplings of the CC can mix quarks/leptons of different families, for instance the probability amplitude for $W_\mu^+\to u\bar s$ is proportional to $V_{us}$, which in general is non-zero. 

Matrices $\ckm$ and $\pmns$ are general unitary matrices, so they can be complex and we must be careful when we write down Feynman rules, since in general $V_{ij}\neq V_{ij}^*$. 


\section{Counting of physical parameters in the Yukawa sector}

Take for instance the quark sector, an analogous discussion holds also for the leptonic sector. In the Yukawa Lagrangian in the flavor basis we introduced complex matrices $Y_U$ and $Y_D$. The number of parameters of 2 generic complex $3\times 3$ matrices is $2\times 2N^2=36$. However not all of these degrees of freedom are physical. To count the number of physical parameters one has to go to the mass basis. 

The number of physical parameters in the Yukawa Lagrangian is given by $M_U^{\text{diag}}=\text{diag}(m_u,m_c,m_t)$ and $M_D^{\text{diag}}=\text{diag}(m_d,m_s,m_b)$, i.e. we have 3+3=6 degrees of freedom. Moreover we have d.o.f. related to $\ckm$ in the charged sector, in particular a generic $3\times3$ unitary matrix contains $\frac12N(N-1)=3$ free moduli and $\frac12N(N+1)=6$ free phases. 

Not all the phases in the $\ckm$ are physical, in the sense that one has always the freedom to redefine (complex) spinor by an arbitrary phase (indeed $U\bar U$ is invariant under such redefinition).
Let's redefine the $U$ and $D$ triplets as
\[\begin{aligned}
\hat U=\begin{pmatrix}e^{i\alpha_1}&0&0\\0&e^{i\alpha_2}&0\\0&0&e^{i\alpha_3}\end{pmatrix}U'=\phi_UU'
\hspace{1.5cm}
\hat D=\begin{pmatrix}e^{i\beta_1}&0&0\\0&e^{i\beta_2}&0\\0&0&e^{i\beta_3}\end{pmatrix}D'=\phi_DD'
\end{aligned}\]
Notice that for each spinor we have 3 d.o.f. in the choice of the phases. The neutral sector is invariant under these redefinition, hence for the neutral sector this redefinition is actually a $U(1)^6$ symmetry transformation. On the other side, the charged sector is not invariant under this $U(1)^6$ transformation (e.g. it mixes up and down quarks)
\[\bar U_L'\gamma^\mu\ckm D_L'\quad\to\quad\hat{\bar U}_L'\gamma^\mu (\phi_U\ckm\phi_D^\dagger)\hat D_L'\]
This phases redefinition modify the $\ckm$ matrix
\[\ckm\quad\to\quad\hat V_{\text{CKM}}\equiv\phi_U\ckm\phi_D^\dagger\]
and one can choose the $\phi_U$ and $\phi_D$ phases in such a way to cancel some of the $\ckm$ phases. Such cancellation reduces the number of degrees of freedom in $\ckm$.  

Note that the charged sector (and all the EW Lagrangian) is invariant under the following global $U(1)_B$ transformation:
\[U'\quad\to\quad\hat U'=e^{i\alpha}U'
\hspace{1.5cm}
D'\quad\to\quad\hat D'=e^{i\alpha}D'\]
This $U(1)_B$ symmetry\footnote{B = Baryon} is the one associated to the baryon number conservation. In particular all quarks and antiquarks have $1/3$ and $-1/3$ $U(1)_B$ charge respectively.

Due to the $U(1)_B$ global symmetry 1 phase is arbitrary, i.e. only $2N-1=5$ phases can be removed by spinorial phase redefinition. 

Finally, the CKM matrix has $\frac12N(N-1)=3$ d.o.f. in moduli and $\frac12N(N+1)-(2N-1)=1$ d.o.f. in the phase. 
Then number of independent parameters in the Yukawa (quark) sector is then 
\[\begin{cases}\begin{matrix}
2N&\text{masses}&\to&6\\
\frac12N(N-1)&\text{moduli}&\to&3\\
\frac12(N-1)(N-2)&\text{phases}&\to&1
\end{matrix}\end{cases}
\qquad\Longrightarrow\qquad 10\text{ independent parameters}
\]
Hence there are 10 independent parameters in the quark sector, instead of the 36 ones we discussed before. 
With the presence of $\nu_R$ there are 10 independent parameters also in the leptonic sector.

The fact that $\ckm$ and $\pmns$ are complex implies that SM violates  $CP$ symmetry. The only discrete symmetry of SM is CPT. 








\end{document}